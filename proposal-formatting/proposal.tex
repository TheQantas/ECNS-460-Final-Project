\documentclass{article}

\usepackage{hyperref}

\title{ECNS 460 Project Proposal}
\author{Quinlin Gregg}
\date{\today}

\begin{document}

\maketitle

\section{Introduction}
The American Gaming Association, a trade group representing casinos and other gambling institutions,
projects that \$35 billion will be bet on the 2024 National Football League (NFL) season \cite{nflbettingamount}.
The sports betting market in the US makes \$14.30 billion in revenue as of 2024,
and is expected to grow to \$23.80 billion by 2029.
Of this, \$2.8 billion (19.6\%) was on American football in 2024 \cite{sportsbettingmarketsize}.

A spread bet is a type of bet where the bettor bets whether the team favored to win will win by at least
\textit{n} points, where \textit{n} is a margin set by the sportsbook. A team's performance Against the
Spread (ATS) is defined as the percentage of games where a favored team wins by at least \textit{n}
points or an underdog team loses by fewer than \textit{n} points.

\section{Research Question}
Does a team's rank in offensive yards and/or points correlate to their performance Against the Spread?
What about in defensive yards and/or points?

\section{Methods}
Data on season-level offensive and defensive statistical performance of NFL teams can be pulled
from sites like \href{https://www.pro-football-reference.com}{Pro Football Reference}.
Historical betting odds can be pulled from various aggregation sites that archive historical odds
from various sportsbooks. This project is looking to use American or Australian sportsbooks.
The advantage of American sportsbooks is that they are more relevant to a United States-centric
NFL audience, but legal sports betting is fairly new in the United States leading to limited historical
data. By contrast, sports betting has been legal in Australia for much longer, meaning there is
much more historical data.

The game-level betting odds will be converted
to season-level by aggregating ATS performance over the season. These season-level ATS values will be
merged into the season-level statistical performance data. The ranks of team teams in all relevant
statistical categories will be calculated to remove time trends. The ranks will then be regressed
with the ATS values to find any relevant relationships.

The results will be presented alongside a Shiny project to make the results more accessible to
those who may not have R or other statistical tools. The site could allow the user to gain familiarity
with trends in NFL statistics as well as explore how each team has performed historically ATS relative
to expectations based on their statistical ranks.

\bibliographystyle{IEEEtran}
\bibliography{references}

\end{document}
